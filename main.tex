\documentclass[fleqn,11pt]{olplainarticle}
% Use option lineno for line numbers 

\title{Brain - Computer Interfaces - Study Case for Industry Project}

\author[1]{Arturs Elksnis}

\keywords{Keyword1, Keyword2, Keyword3}

\begin{abstract}
Please provide an abstract of no more than 300 words. Your abstract should explain the main contributions of your article, and should not contain any material that is not included in the main text. 
\end{abstract}

\begin{document}

\flushbottom
\maketitle
\thispagestyle{empty}

\section*{Introduction}
Breaking the communication bottleneck by adding additional communication channels from the brain to the computer could have profound implications on the way we interact with and benefit from the computer. Additional information may increase the overall communication rate and thus could provide a mechanism to increase human efficiency. Alternatively, augmented awareness about the current state of the brain could make interaction with computers a more natural experience that in the end may not differ from the way we interact with and experience our own body. For example, we might simply focus attention to an Internet link to follow it rather than producing complicated motor commands to move and click a mouse, or we might merely feel that a particular menu selection is not appropriate rather than having to learn the same by reading text on a screen. In summary, the processes that transform our intent into the actions necessary to achieve it could become simpler if we had better access to the current state of the brain. \cite{schalk2008brain}

With relatively modest improvements, brain-computer interfacing technology will become a practical and safe, albeit simple and slow, communication aid. It will thus soon become of interest to the first group of adopters: handicapped individuals who are currently limited for essentially all tasks by their limited communication capacity. For these people, even the modest rates of communication that will initially be achieved should dramatically improve quality of life.\cite{schalk2008brain}

If it becomes possible to design an (ideally non-invasive) interface (see Section 3.3) that can support high performance at an affordable price, brain-computer interfacing technologies will become of interest to the third group of users – most other members of society – that could use these technologies for a wide variety of purposes. At the same time, this new communication capacity will constitute a radical and disruptive innovation that will not be immediately compatible with existing practice and that will evoke change in many complementary processes. It will thus take some time, perhaps a few decades, until this technology has been fully integrated in human societies.

In summary, I expect that, as performance increases and price decreases, brain- computer interfacing technology will become beneficial to an increasing number of individuals, that the direct and indirect effects of its use will become increasingly pervasive, and that the implications on individuals and society will grow in parallel. I thus anticipate that this development of brain-computer interfacing technology will in many ways mirror the development of computers (that addressed the previous bottleneck in human productivity) and of other General-Purpose Technologies (GPTs) [148]. GPTs have been found to have a wide variety of major effects on private and social performance [149]. For example, Information Technology and the Internet have wide applications and productivity-enhancing effects in numerous downstream sectors with high social rates of return that often exceed private rates of return [150, 151], and their dissemination is having a sustained, long-lasting impact on productivity and economic growth. Brain-computer interfacing technology can thus be expected to have a similar profound impact not only on individual, but also on societal performance.
\cite{schalk2008brain}

At the same time, the full potential of direct brain-to-computer communication will only be realized when this technology can benefit most members of society. As soon as interfaces can be built that can interface safely, economically, and concurrently with most of the major systems in the brain, many applications will emerge that will augment our senses and our communication capacities with others and with computers. It will be then that enhanced communication capacities will pervade the fabric of society with a multitude of side effects on many other technologies and processes.\cite{schalk2008brain}

Privacy and Liability in \cite{schalk2008brain}.

\section*{Methods and Materials}

Guidelines can be included for standard research article sections, such as this one.

\section*{Some \LaTeX{} Examples}
\label{sec:examples}

Use section and subsection commands to organize your document. \LaTeX{} handles all the formatting and numbering automatically. Use ref and label commands for cross-references.

\subsection*{Figures and Tables}

Use the table and tabular commands for basic tables --- see Table~\ref{tab:widgets}, for example. You can upload a figure (JPEG, PNG or PDF) using the project menu. To include it in your document, use the includegraphics command as in the code for Figure~\ref{fig:view} below.

\begin{figure}[ht]
\centering
\includegraphics[width=0.7\linewidth]{frog}
\caption{An example image of a frog.}
\label{fig:view}
\end{figure}

\begin{table}[ht]
\centering
\begin{tabular}{l|r}
Item & Quantity \\\hline
Candles & 4 \\
Fork handles & ?  
\end{tabular}
\caption{\label{tab:widgets}An example table.}
\end{table}

\subsection*{Citations}

LaTeX formats citations and references automatically using the bibliography records in your .bib file, which you can edit via the project menu. Use the cite command for an inline citation, like \cite{lees2010theoretical}, and the citep command for a citation in parentheses \citep{lees2010theoretical}.

\subsection*{Mathematics}

\LaTeX{} is great at typesetting mathematics. Let $X_1, X_2, \ldots, X_n$ be a sequence of independent and identically distributed random variables with $\text{E}[X_i] = \mu$ and $\text{Var}[X_i] = \sigma^2 < \infty$, and let
$$S_n = \frac{X_1 + X_2 + \cdots + X_n}{n}
      = \frac{1}{n}\sum_{i}^{n} X_i$$
denote their mean. Then as $n$ approaches infinity, the random variables $\sqrt{n}(S_n - \mu)$ converge in distribution to a normal $\mathcal{N}(0, \sigma^2)$.

\subsection*{Lists}

You can make lists with automatic numbering \dots

\begin{enumerate}[noitemsep] 
\item Like this,
\item and like this.
\end{enumerate}
\dots or bullet points \dots
\begin{itemize}[noitemsep] 
\item Like this,
\item and like this.
\end{itemize}
\dots or with words and descriptions \dots
\begin{description}
\item[Word] Definition
\item[Concept] Explanation
\item[Idea] Text
\end{description}

\section*{Acknowledgments}

Additional information can be given in the template, such as to not include funder information in the acknowledgments section.

\bibliography{sample}

\end{document}