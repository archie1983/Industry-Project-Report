\documentclass[fleqn,11pt]{olplainarticle}
% Use option lineno for line numbers 

\usepackage{csquotes}
%\usepackage[style=verbose-ibid,backend=bibtex]{biblatex}
\renewcommand{\mkbegdispquote}[2]{\itshape}
\bibliography{sample}

\title{Brain - Computer Interfaces - Study Case for Industry Project}

\author[1]{Arturs Elksnis}

\keywords{Brain, Brain - Computer Interface, BCI}

\begin{abstract}
Please provide an abstract of no more than 300 words. Your abstract should explain the main contributions of your article, and should not contain any material that is not included in the main text. 
\end{abstract}

\begin{document}

\flushbottom
\maketitle
\thispagestyle{empty}

\section{Introduction}
The idea of a Brain - Computer interface (BCI) is decades old, however the latest developments in the field have both opened new possibilities for neurorehabilitation of motor and sensory disabilities and inspired many to imagine applications for healthy people. E.g., the new Neuralink BCI \cite{musk2019integrated} employs more than 1000 electrodes, which closely compares to a hypothetical device in \cite{schalk2008brain} little more than a decade ago. \cite{schalk2008brain} discusses the BCI as an improvement to human efficiency, moreover it talks about much more natural interaction between a machine and a human to the point where the machine feels like an extension of one's body.

While \cite{musk2019integrated} demonstrates a viable engineering solution to fulfill at least some of the speculations in \cite{schalk2008brain} and many of other publications focus on technology aspects of BCI, not enough attention has been given to the social implications of the technology and its emotional impact. After all a technology that promises to extend our physical capabilities must introduce profound changes to our social fabric, if not change the whole human experience- much like the introduction of Internet and mobile phones did. \cite{kogel2019using} supports the sentiment of needing more social research 
to understand BCI:
\begin{displayquote}
    To understand BCI beyond its technical components and medical applications, significant social research is needed to grasp BCI in its practical and human dimension. Insights into the use of BCI and its impact on the user are necessary to develop the relevant knowledge and tools for ethical and legal evaluation.
\end{displayquote}

This report therefore tries to identify and analyse potential social and emotional impact of BCI on human society by looking at the issues experienced by current users of BCIs and the needs of potential users. Some parallels are drawn between BCI and social networks and some technological co-development of the two is speculated on.

\section{Methodology}
The research method employed in this report is purely literature review due to the defined scope of the report. A review of current social issues of BCI use is provided, followed by a few ideas of potential applications. The potential BCI development trajectory is speculated on based on the development of Internet and particularly social networks. Finally, several social issues most important for wide acceptance of BCI are identified along with recommendations to address them in section \ref{sec:results}.

The PubMed database of  National Center for Biotechnology Information (NCBI) was used as a source for most of the publications and articles reviewed.

\section{Discussion}
\cite{schalk2008brain} predicts that the first group of BCI adopters would be people with medical needs which can be adressed partially or fully with a BCI. This would then be followed by military use where BCI would improve the capabilities of personnel. Finally BCI would be adopted by healthy people outside military. Similarly, albeit in a different context, \cite{kogel2019using} distinguishes two groups of current BCI users- physically impaired and healthy ones. This distinction is also extended to potential BCI users, who have no prior direct experience with BCI. It therefore feels logical to organise this discussion section with the same kind of BCI users' distinction- the impaired and the healthy ones.

\subsection{Physically Impaired Users}
According to \cite{kogel2019using} physically impaired users generally see BCI as a positive enabling technology:
\begin{displayquote}
    Those affected by physical impairment tend to hope for more independence and social participation and expect from BCI use an increased quality of life. Some studies recognized the potential of BCI use to contribute to the user’s self-esteem and self-expression. BCIs are reported as bringing satisfaction and enjoyment, although space for improvement certainly remains in this area.
\end{displayquote}

Among the concerns the following are mentioned in \cite{kogel2019using}:
\begin{displayquote}
    \begin{itemize}
        \item Worries regarding surplus-work effort for caregivers, but also provides caregivers with more time while using BCI.
        \item Mental and physical fatigue, anxiety, pain/discomfort.
        \item With implant: feeling self-conscious, irritation about usage; difficulty of control.
        \item Cap, electrodes, frustration about BCI illiteracy.
        \item Discomfort and annoyance (preparations and electrodes), burden of transportation, fatigue, disappointment/anger (about failure).
        \item Increased dependence on others.
        \item Set-up time, cap (comfort and look), need for washing hair after training, limited mobility, low speed.
        \item Physical and mental strains, frustration, belied expectations, pain.
    \end{itemize}
\end{displayquote}

The following expectations by potential users are mentioned in \cite{kogel2019using}:
\begin{itemize}
    \item Ease of use, high performance, little maintenance, decent aesthetics.
    \item independent use, convenient use, non-invasiveness, functions, costs, set-up time
    \item accuracy, speed, simplicity, standby mode
    \item better speed and set-up time
    \item knowledge about risks/rewards
    \item various control functions (TV, bed, emergency alarm)
    \item improve autonomy, home use, ease of use, be light and wearable
    \item 	functionality, independence (mobility, daily life activities, employment, ease of use)
\end{itemize}

The non-impaired current users' expectations according to \cite{kogel2019using} are:
\begin{itemize}
    \item more aesthetic and practical cap, integration of other devices and entertainment system, improvements in terms of handling difficulty and graphics
\end{itemize}

They report "experiences of transparency of activities, sense of “presence” in VR, imagination of avatar as being (part of) themselves".

The professionals seem to be concerned about the following:
\begin{itemize}
    \item ethical concerns reported: the duty of correct information transfer, avoiding unrealistic expectations in participants, BCI illiteracy, the risk of detrimental brain modifications due to BCI use and privacy issues
    \item disagreement regarding terminology/definitions of BCIs and marketability of different BCIs; ethical concerns reported: informed consent, benefits/risks, team responsibility, consequences, liability/personal identity, and interaction with the media; non-invasive BCIs are estimated as being of low risk (indecisive about invasive BCIs); most BCI professionals hold the view that BCI users are responsible for their actions, while being uncertain regarding issues of liability; the effect of BCI activity on personal identity and self-image on the users are deemed to be unclear
\end{itemize}

The points addressed by caregivers according to \cite{kogel2019using}:
\begin{displayquote}
    \begin{itemize}
        \item BCI is regarded as an opportunity to maintain communication between caregivers and caretakers; caregivers would appreciate the opportunity of “back communication” (i.e. informing their caretakers, e.g. letting them know that they are on their way); caregivers also see an additional burden in dealing with the BCI
        \item caregivers ranked BCI functions similar to their caretakers: priority of accuracy, variety of functions, standby reliability, wheelchair and computer control
        \item reported expectations towards BCIs: information about BCIs and their applications, a system that adapts to the various stages of the disease, taking account of emotion, and retaining the users’ sense of agency
    \end{itemize}
\end{displayquote}

-> Give summary of the above.

-> Cochlear implants.

-> Advertising directly to brain.

-> Who's responsible for involuntary or transsient thoughts brought to life by machines?

-> Transfer intuition or even knowledge from professor to anyone who pays

-> Animals rights activists

-> BCI Inbound and Outbound communication split. Regulations for either.

\subsection{Potential applications}
Suppose that BCI technology can be quickly developed further. We already have claims of developments like the capability of streaming music directly to the brain in \cite{pero_2020}. We also have a demonstration of a primate using a BCI to control a computer game in \cite{wakefield_2020}. These are significant developments towards:
\begin{enumerate}
    \item BCI extracting information from the brain. Let's call this "reading from the brain".
    \item BCI actively stimulating the brain. Let's call this "writing to the brain".
\end{enumerate}
It's easy to see how this technology can be used for both good and evil. 

It could change the way we think about studying, e.g. potentially a student could just copy a professor's intuition if not their whole subject expertise instead of spending time and effort to learn it themselves. While this would allow for incredible human potential gains, a question must be asked, what then becomes of the value of an academic degree? Perhaps subject expertise could be provided as a service or commodity. This would allow for a very efficient knowledge sharing, but then who determines the price of such service or commodity and based on what?

It could have much darker connotations too. E.g. the endorphins' release could be controlled by the amount of advertisements we've seen or accepted directly into our brains. Perhaps dreaming in sleep or even sleep itself could only become available if paid for. 

BCI could even lead to a hive mind where an individual's role would be more akin to that of a worker ant than what we're used to presently.

While some of these fears are reflected in \cite{liberatore_2021}, I am of opinion that for BCI technology to be dangerous enough, it will require the "writing to the brain" functionality to be much more mature than it is now. The "reading from the brain" functionality however seems to be closer to commercialisation, therefore let's explore a speculative, but conceivably realistic potential BCI that would effectively replace a computer keyboard and allow the end user much faster typing.

\subsubsection{BCI as a computer keyboard replacement}
Initially this kind of BCI would likely accelerate the interaction speed of human brain with a computer bringing productivity improvements and time savings. While this kind of interaction with the computer could feel unnatural at first, so would have been the first attempts to use a keyboard that we're all too used to by now. Text input from online forms to scientific papers would then be limited by the speed of thought rather than fingers. Perhaps mobile devices with touch screen would become much more convenient to use with most of the typos gone. That in turn could lead to these devices be used for work a lot more, which would in a sense allow to carry our office with us. While that would improve the mobility of workforce and the sense of freedom by the end user, the expectation for the worker to be available 24/7 would need to be managed more efficiently than now. The increased productivity could conceivably create job losses. Perhaps an issue of typing while driving would become even more acute than now due to the split attention. These problems are not new, but they would be exacerbated by this hypothetical BCI along with the benefits of it. 

\subsubsection{BCI leading to human fusion with machines}
Ultimately though BCI is a step towards human fusion with the machine.

\subsection{Comparison with the development of Internet}
It is clear that BCI has the potential to significantly change our lives. It even feels like one of the key technologies for us to become a type 1 civilisation in the extended Kardashev scale \cite{gray2020extended}. Also perhaps we can make some educated guesses about its development trajectory if we compare it to another civilisation-advancing technology- the Internet and its development.

According to \cite{leiner2009brief}: "by the end of 1969, four host computers were
connected together into the initial ARPANET, and the budding Internet was off the ground". Merely 3 years later in 1972, the first e-mail application was developed which became the largest network application for over a decade and was the beginning of people-to-people communication that defines Internet today. The progress between first network and first email was rather rapid and was enabled by the then new concept of packet switching which was used instead of circuit switching to establish a communication channel between two nodes.

If we wanted to draw technology related parallels with BCI, we could compare the advent of Neuralink's new BCI in \cite{musk2019integrated} to the moment in 1969 when the first 4 machines were connected into ARPANET. Perhaps the primate controlling a video game using a BCI in \cite{wakefield_2020} is comparable to the moment of first email application in 1972. And like Internet became possible due to the new concept of packet switching vs circuit switching, the new era of BCI is now possible due to the capability to embed thousands of electrodes directly into the brain tissue as opposed to using less capable non-invasive techniques.

Drawing further technology related parallels however is out of the scope of this paper and we'll focus instead on exploring aspects of human nature that made Internet part of our every day lives and ask if the same aspects can make BCI part of our every day lives.

It's hard to overestimate the impact of people-to-people communication on our lives that is afforded by the internet. Social networks have become an integral part of many people's lives. So what does make them so popular?

According to \cite{cheung2009understanding}: "intention to continue using virtual community is determined by satisfaction, commitment and group norms. These constructs are then affected by the needs of using a virtual community (purposive value, self-discovery, entertainment value, social enhancement, and maintaining interpersonal interconnectivity)."

Same authors in \cite{cheung2011online} give a more concise description for each of these needs: 
\begin{displayquote}
    \begin{itemize}
        \item Purposive value refers to the value derived from accomplishing some pre-determined informational and instrumental purpose.
        \item Self-discovery refers to the understanding and deepening salient aspects of one’s self through social interactions.
        \item Maintaining interpersonal interconnectivity refers to the social benefits derived from establishing and maintaining contact with other people such as social support, friendship, and intimacy.
        \item Social enhancement refers to the value that a participant derives from gaining acceptance and approval of other members, and the enhancement of one’s social status within the community on account of one’s contribution to it.
        \item Entertainment value refers to fun and relaxation through playing or otherwise interacting with others.
    \end{itemize}
\end{displayquote}

It would be reasonable to expect that for BCI to enjoy similar success to social networks, it has to satisfy at least some of these needs. To achieve that, BCI could become an extension to social networks- it's easy to see how the interpersonal interconnectivity can be enhanced through the use of BCI- the experiences and the feeling of presence can be made more intense for remote participants. Likewise, the entertainment value can be enhanced for the same reasons. 

The need of purposive value, however, seems to be better addressed with tele-operation techniques, which together with BCI could conceivably complement each other. E.g., the approach in \cite{luo2019teleoperation} could probably benefit from drawing the human intention directly from a remote BCI instead of using a haptic device and reading electromyography (EMG) signals.

\cite{shih2017deep} demonstrates a reinforcement learning setup that takes user's emotions as a learning feedback. This method allows for a tight coupling between a human and a machine on an emotional level. Also the direct access to the end user's emotions offers a significantly advantageous opportunity to directly tune the experience thus improving it.

\cite{kogel2019using} gives a useful summary of 73 publications that touch upon social aspects of BCI.

Some were concerned about data protection and abuse. Additional concerns raised include the creation of self-transcending human-machine hybrids, as well as the fear for further dependencies, as the BCI necessitates service support and technological maintenance.\cite{kogel2019using} 

Some study participants expected an improved level of self-expression while others feared a distortion of their sense of self. The new brain stimulation device might also further restrain the user’s sense of accountability, in the case that others would hold the device responsible for their behavior or expressions. Furthermore, this application would require more trust in researchers to keep their data secure and confidential. This last point concerns the issue of meaningful consent.\cite{kogel2019using} 

BCIs can benefit individuals by restoring varying degrees of motor control and possibly the ability to communicate. But expectations of some subjects and their caregivers may exceed what they can reasonably achieve with the technology and result in psychological harm. Selecting candidates with higher levels of preserved cognitive function for BCI research and treatment and educating them on the potential benefits and limitations of the technique is advisable to prevent or at least minimize harm.\cite{glannon2014ethical}

Moreover, Fernandez-Espejo and Owen acknowledge that, with current interface technology, simple affirmative or negative responses to questions about whether a minimally conscious patient wanted to continue living would not be sufficient to establish that the patient had the “cognitive and emotional capacity to make such a complex decision” (Fernandez-Espejo and Owen, 2013, p. 808). But they also say that “it is only a matter of time before all of these obstacles are overcome” (p. 808).\cite{glannon2014ethical}

\cite{sexton2015overlooked} talks about brain-to-brain coupling and suggests that BCI efficiency increases with increased user interaction with each other as opposed to using the BCI in isolation.

\cite{kogel2020like} reviews experiences of patients who use BCI to restore their bodily functions. Positive

\cite{kawala2021summary} Sheds some light on the BCI aspects that are most important to the users. Unsurprisingly, the BCI safety is the single most important aspect.

gives some insight into the social aspects of the patient-caregiver interaction in medical use cases of BCI.

\section{Results}
\label{sec:results}
The idea of BCI controlled closed-loop brain stimulation is met with acceptance among some potential users, while others feel ambivalent or opposed to the concept in principle. Comparing this hypothetical setting with the experiences of the study participants with open-loop brain stimulation, some participants welcomed the prospect of a self-controlled brain stimulation device while other participants maintained reservations. Some study participants expected an improved level of self-expression while others feared a distortion of their sense of self. The new brain stimulation device might also further restrain the user’s sense of accountability, in the case that others would hold the device responsible for their behavior or expressions. Furthermore, this application would require more trust in researchers to keep their data secure and confidential. This last point concerns the issue of meaningful consent. Participants mentioned difficulty in understanding all risks and implications of a closed-loop device both for themselves and for individuals with cognitive impairments. \cite{kogel2019using}

Participants hoped for more independence, self-control, privacy/intimacy, and better communication. Some were concerned about data protection and abuse. Additional concerns raised include the creation of self-transcending human-machine hybrids, as well as the fear for further dependencies, as the BCI necessitates service support and technological maintenance. \cite{kogel2019using}

Breaking the communication bottleneck by adding additional communication channels from the brain to the computer could have profound implications on the way we interact with and benefit from the computer. Additional information may increase the overall communication rate and thus could provide a mechanism to increase human efficiency. Alternatively, augmented awareness about the current state of the brain could make interaction with computers a more natural experience that in the end may not differ from the way we interact with and experience our own body. For example, we might simply focus attention to an Internet link to follow it rather than producing complicated motor commands to move and click a mouse, or we might merely feel that a particular menu selection is not appropriate rather than having to learn the same by reading text on a screen. In summary, the processes that transform our intent into the actions necessary to achieve it could become simpler if we had better access to the current state of the brain. \cite{schalk2008brain}

With relatively modest improvements, brain-computer interfacing technology will become a practical and safe, albeit simple and slow, communication aid. It will thus soon become of interest to the first group of adopters: handicapped individuals who are currently limited for essentially all tasks by their limited communication capacity. For these people, even the modest rates of communication that will initially be achieved should dramatically improve quality of life.\cite{schalk2008brain}

If it becomes possible to design an (ideally non-invasive) interface (see Section 3.3) that can support high performance at an affordable price, brain-computer interfacing technologies will become of interest to the third group of users – most other members of society – that could use these technologies for a wide variety of purposes. At the same time, this new communication capacity will constitute a radical and disruptive innovation that will not be immediately compatible with existing practice and that will evoke change in many complementary processes. It will thus take some time, perhaps a few decades, until this technology has been fully integrated in human societies.

In summary, I expect that, as performance increases and price decreases, brain- computer interfacing technology will become beneficial to an increasing number of individuals, that the direct and indirect effects of its use will become increasingly pervasive, and that the implications on individuals and society will grow in parallel. I thus anticipate that this development of brain-computer interfacing technology will in many ways mirror the development of computers (that addressed the previous bottleneck in human productivity) and of other General-Purpose Technologies (GPTs) [148]. GPTs have been found to have a wide variety of major effects on private and social performance [149]. For example, Information Technology and the Internet have wide applications and productivity-enhancing effects in numerous downstream sectors with high social rates of return that often exceed private rates of return [150, 151], and their dissemination is having a sustained, long-lasting impact on productivity and economic growth. Brain-computer interfacing technology can thus be expected to have a similar profound impact not only on individual, but also on societal performance.
\cite{schalk2008brain}

At the same time, the full potential of direct brain-to-computer communication will only be realized when this technology can benefit most members of society. As soon as interfaces can be built that can interface safely, economically, and concurrently with most of the major systems in the brain, many applications will emerge that will augment our senses and our communication capacities with others and with computers. It will be then that enhanced communication capacities will pervade the fabric of society with a multitude of side effects on many other technologies and processes.\cite{schalk2008brain}

Privacy and Liability in \cite{schalk2008brain}.

\printbibliography
\end{document}