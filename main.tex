\documentclass[fleqn,11pt]{olplainarticle}
% Use option lineno for line numbers 

\usepackage{csquotes}
%\usepackage[style=verbose-ibid,backend=bibtex]{biblatex}
\renewcommand{\mkbegdispquote}[2]{\itshape}
\bibliography{sample}

\title{Brain - Computer Interfaces - Study Case of Social Impact}

\author[1]{Arturs Elksnis}

\keywords{Brain, Brain - Computer Interface, BCI, Social impact}

\begin{abstract}
Please provide an abstract of no more than 300 words. Your abstract should explain the main contributions of your article, and should not contain any material that is not included in the main text. 
\end{abstract}

\begin{document}

\flushbottom
\maketitle
\thispagestyle{empty}

\section{Introduction}
The idea of a Brain - Computer interface (BCI) is decades old, however the latest developments in the field have both opened new possibilities for neuro-rehabilitation of motor and sensory disabilities and inspired many to imagine applications for healthy people. E.g., the new Neuralink BCI \cite{musk2019integrated} employs more than 1000 electrodes, which closely compares to a hypothetical device in \cite{schalk2008brain} little more than a decade ago. Interestingly \cite{schalk2008brain} discusses the BCI as an improvement to human efficiency, moreover it talks about much more natural interaction between a machine and a human to the point where the machine feels like an extension to one's body.

While \cite{musk2019integrated} demonstrates a viable engineering solution to fulfill at least some of the speculations in \cite{schalk2008brain} and many other publications focus on technology aspects of BCI, not enough attention has been given to the social implications of the technology and its emotional impact. After all a technology that promises to extend our physical capabilities must introduce profound changes to our social fabric- much like the introduction of Internet and mobile phones did. The importance of BCI social aspects is supported in articles like \cite{sexton2015overlooked}, \cite{kogel2019using} and others.

This report therefore tries to identify and analyse potential social and emotional impact of BCI on human society by looking at the issues experienced by current users of BCIs and the needs of potential users. Some parallels are drawn between BCI and social networks and some technological co-development of the two is speculated on.

\section{Methodology}
The research method employed in this report is purely literature review due to the defined scope of the report. A thought experiment about several potential BCI applications is explored, followed by a review of current acceptance of BCIs. The potential BCI development trajectory is speculated on based on the development of Internet and particularly social networks. Finally, several social issues most important for wide acceptance of BCI are identified along with recommendations to address them in section \ref{sec:results}.

The PubMed database of  National Center for Biotechnology Information (NCBI) was used as a source for most of the publications and articles reviewed.

\section{Results}
For the purposes of assessing the social impact of the BCI, it is important to separate BCI functionality into two categories: BCI reading the brain and BCI writing to the brain. While there is more demonstrable evidence of the former, it is the latter that is expected to bring the biggest changes to our social fabric. 

There is evidence that public's attitude to BCIs change significantly depending on the purposes, that BCI is used for \cite{meyer2018disabled} with neuro-rehabilitation being seen much more positively than body augmentation in healthy individuals. In the short term this attitude should limit BCI use outside of people requiring neuro-rehabilitation. In the longer term I expect the BCI to be used as a means of extending our biological capabilities \cite{warwick2003cyborg}. Perhaps this will coincide with the maturing of the "writing to the brain" functionality.

The advances of BCI technology raise many important questions about the ethics and social impact of BCI. E.g.:
\begin{enumerate}
    \item In the context of BCI as a means of controlling a machine, who will be liable for wrongful or inadvertent interpretation and execution of a BCI user's thought?
    \item In the context of BCI writing to the brain, how can the user be protected from unethical commercial practices?
    \item In the context of BCI writing to the brain, how can the user's autonomy be protected if their decisions can be directly influenced by external stimuli?
    \item With the potential rise of cyborgs, how can the protection and indeed the preservation of the rest of humanity be guaranteed?
    \item Given the emotional component of human nature and the inevitable irrationality that it often brings, how can the machine be prevented from exploiting this vulnerability, when the BCI can measure user's emotions? 
\end{enumerate}

\section{Discussion}
Suppose that BCI technology can be quickly developed further. We already have claims of developments like the capability of streaming music directly to the brain in \cite{pero_2020}. We also have a demonstration of a primate using a BCI to control a computer game in \cite{wakefield_2020}. These are significant developments towards:
\begin{enumerate}
    \item BCI extracting information from the brain. Let's call this "reading from the brain".
    \item BCI actively stimulating the brain. Let's call this "writing to the brain".
\end{enumerate}

Sharing the excitement about BCI in \cite{warwick2003cyborg} and \cite{schalk2008brain}, it is easy to imagine how the rapidly maturing BCI technology could change our lives. So let's start this discussion with a simple thought experiment about several hypothetical, but conceivably realistic BCIs.

\subsection{Reading from the brain only}
\subsubsection{BCI as a computer keyboard replacement}
A BCI, that allowed direct brain-to machine communication by reading the brain and writing to the machine, would likely accelerate information transfer to the machine \cite{schalk2008brain} bringing productivity improvements and time savings, but the impact would probably be not too different from the impact of smart mobile phones, in fact it would probably be an extension to them. Perhaps mobile devices with touch screen would become much more convenient with regards to typing speed and effort, which would lead to these devices being used for work a lot more. While that would improve the mobility and flexibility of workforce, the expectation for the worker to be available 24/7 would become even more prevalent. The increased productivity could conceivably create job losses. Perhaps typing while driving would become a bigger issue. These problems are not new, but they would be exacerbated by this hypothetical BCI along with the benefits of it. People requiring neuro-rehabilitation would benefit, but on balance the social impact would probably be somewhat limited. Perhaps BCI would feel unnatural at first, but then so would have felt the first keyboards, which we're all too used to by now.

\subsubsection{BCI for remote control of machines}
The computer keyboard replacement BCI could of course be used for remote control of machines too, but \cite{schalk2008brain} suggests and \cite{warwick2003cyborg} and \cite{wakefield_2020} demonstrate approach where the machine is controlled in a more natural way with BCI interpreting the user's intent and sending off the required commands to the machine. While it can effectively extend human body capabilities beyond its biological boundaries and make it feel like a natural extension, issues may arise with liability for incorrectly detected human intention. E.g., if the BCI misinterprets the human intention or enacts an involuntary or transient thought, which was not really meant to be enacted, the end result can be disastrous. \cite{schalk2008brain} raises the question of whose fault would it be- the user's or the machine's? 

\subsection{BCI with mature "writing to the brain" functionality}
A lot more interesting albeit seemingly harder to achieve would be the functionality to write to the brain. The social implications in this case would be much more profound, in my opinion, as it would open much more exotic opportunities to what we're used or indeed able to expect. E.g., it could change the way we think about studying, i.e. potentially a student could just purchase a copy of professor's intuition if not their whole subject expertise for their own BCI instead of spending time and effort to learn it themselves. While this would allow for incredible human potential gains, what then becomes of the value of an academic degree? Perhaps subject expertise could be provided as a service or commodity. This would allow for a very efficient knowledge sharing, but then who determines the price of such service or commodity and based on what? For an individual- would it lead to high specialisation in narrow subject matters as it has happened historically or would it give us universal access to knowledge that is as broad as it is deep? 

There are concerns that if our brains are stimulated by BCI, then our decisions may become not really ours as a result \cite{kogel2019using}. This however is a philosophical question- after all almost all new technology affects our decision making processes- e.g., invention of the wheel may have changed our decisions on what we choose to take with us or leave behind when travelling. Never before however the technology has had the potential to inject thoughts into our minds directly.

I expect much darker connotations too. E.g. the endorphins' release could be controlled by the amount of advertisements we have seen or accepted directly into our brains. The BCI user could be made addicted to any commercial product.

Perhaps dreaming in sleep or even sleep itself could only become available if paid for. 

BCI could lead to a hive mind where an individual's role would be more akin to that of a worker ant than what we're used to being presently. Some of these and other fears are reflected in \cite{liberatore_2021}, \cite{kogel2019using} and \cite{warwick2003cyborg}.

\subsection{General Acceptance of BCI}
Ultimately though BCI is a step towards human fusion with the machine. \cite{warwick2003cyborg} lists several potential benefits of human fusion with a machine, in particular harnessing computational power of the machine and expanded perception of the world due to many available sensors. Same publication also recognises fears that different cultures and groups of people may have towards BCI. Alarmingly \cite{warwick2003cyborg} also warns about the potential of cyborgs splitting from human species and making the rest of the humans dominated and obsolete.

As for current situation, there is evidence that society in general does not view kindly those who are able-bodied yet choose to enhance their bodies through bionics \cite {meyer2018disabled}. On the other hand the attitudes towards disabled people using bionics is positive. Furthermore disabled non-users themselves tend to have positive attitude towards BCI with the expectation of their quality of life improvements \cite{kogel2019using}. Likewise the current BCI experiences of disabled people are mostly positive \cite{kogel2020like}, although the expectations at times need to be carefully moderated \cite{glannon2014ethical}.

Perhaps more BCI acceptance among healthy people can be expected once we are more used to the technology, but for now it is reasonable to expect that most first generation of cyborgs will be people who require neuro-rehabilitation. This expectation is supported in \cite{schalk2008brain} and \cite{kogel2019using}.

\subsection{Comparison with the development of Internet}
BCI feels like one of the key technologies for us to become a type 1 civilisation in the extended Kardashev scale \cite{gray2020extended} and perhaps we can make some educated guesses about its development trajectory if we compare it to another civilisation-advancing technology- the Internet and its development.

According to \cite{leiner2009brief}: "by the end of 1969, four host computers were
connected together into the initial ARPANET, and the budding Internet was off the ground". Merely 3 years later in 1972, the first e-mail application was developed which became the largest network application for over a decade and was the beginning of people-to-people communication that defines Internet today. The progress between first network and first email was rather rapid and was enabled by the then new concept of packet switching which was used instead of circuit switching to establish a communication channel between two nodes.

If we wanted to draw technology related parallels with BCI, we could compare the advent of Neuralink's new BCI in \cite{musk2019integrated} to the moment in 1969 when the first 4 machines were connected into ARPANET. Perhaps the primate controlling a video game using a BCI in \cite{wakefield_2020} is comparable to the moment of first email application in 1972. And like Internet became possible due to the new concept of packet switching vs circuit switching, the new era of BCI is now possible due to the capability to embed thousands of electrodes directly into the brain tissue as opposed to using less capable non-invasive techniques.

Drawing further technology related parallels however is out of the scope of this paper, but arguably the most popular aspect of Internet- the social networks are interesting in the context of BCI. Social networks have become an integral part of many people's lives. So what does make them so popular?

According to \cite{cheung2009understanding}: "intention to continue using virtual community is determined by satisfaction, commitment and group norms. These constructs are then affected by the needs of using a virtual community (purposive value, self-discovery, entertainment value, social enhancement, and maintaining interpersonal interconnectivity)."

Same authors in \cite{cheung2011online} give a more concise description for each of these needs: 
\begin{displayquote}
    \begin{itemize}
        \item Purposive value refers to the value derived from accomplishing some pre-determined informational and instrumental purpose.
        \item Self-discovery refers to the understanding and deepening salient aspects of one’s self through social interactions.
        \item Maintaining interpersonal interconnectivity refers to the social benefits derived from establishing and maintaining contact with other people such as social support, friendship, and intimacy.
        \item Social enhancement refers to the value that a participant derives from gaining acceptance and approval of other members, and the enhancement of one’s social status within the community on account of one’s contribution to it.
        \item Entertainment value refers to fun and relaxation through playing or otherwise interacting with others.
    \end{itemize}
\end{displayquote}

It would be reasonable to expect that for BCI to enjoy similar success to social networks, it should satisfy or help satisfy at least some of these needs. To achieve that, BCI could become an extension to social networks- it is easy to see how the interpersonal interconnectivity can be enhanced through the use of BCI- the experiences and the feeling of presence can be made more intense for remote participants. Likewise, the entertainment value can be enhanced for the same reasons. To meet these needs, virtual environments (VE) can be used as shown in \cite{shih2017deep}. The same article demonstrates a reinforcement learning setup that takes user's emotions as a learning feedback. It is easy to envisage how this method can establish tight coupling between a human and a machine on an emotional level. Also the direct access to the end user's emotions offers a significantly advantageous opportunity to directly tune the experience thus improving it. It is not unreasonable to expect that emotion reading by BCI will present commercial opportunities to meet the needs of interpersonal interconnectivity and entertainment. It is also resonable to expect that this will affect our social fabric by making us less empathetic and increasingly self-segregated into various virtual echo-chambers similarly to what prevalence of social networks and mobile phones did to us. After all it is all too easy to segregate people into groups based on their emotional reactions to events around us. On the positive side, this application of BCI will allow us to stay in much closer contact to loved ones far away. From personal experience, Internet applications like WeChat, WhatsApp, etc., can already provide noticeable improvement of the feeling of presence over writing letters or having phone conversations with family members. Perhaps going and living abroad will feel much less of an issue on an emotional level given much improved emotional link to back home. However the emotions themselves present an exploitable vulnerability as they are often the source of irrationality in human behaviour, therefore any use of BCI should be ethical with regards to our emotions. This is further supported by the evidence of emotional attachment to implants in \cite{warwick2003cyborg}.

The need of purposive value, however, seems to be better addressed with tele-operation techniques, which together with BCI could conceivably complement each other. E.g., the approach in \cite{luo2019teleoperation} could probably benefit from drawing the human intention directly from a BCI instead of using a haptic device and reading electromyography (EMG) signals. \cite{warwick2003cyborg} mentions a successful attempt of driving a robot hand with nerve signals that were generated thousands of miles away from it. I expect that teleoperation through BCI will eventually become part of everyday norm allowing us to effectively be in two places at the same time. This will probably change the way we see work- there will be less need for commuting and our productivity will rise as a result. Once again, this can cause joblessness in short term, but will probably allow for a better quality of life.



\subsection{Conclusion}
\cite{sexton2015overlooked} talks about brain-to-brain coupling and suggests that BCI efficiency increases with increased user interaction with each other as opposed to using the BCI in isolation.

\cite{kawala2021summary} Sheds some light on the BCI aspects that are most important to the users. Unsurprisingly, the BCI safety is the single most important aspect.

\printbibliography
\end{document}